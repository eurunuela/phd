\title{Conclusion and future directions}
\label{cha:conclusion}

\section{Summary and Final Discussion}

As mentioned in the introduction (\cref{cha:intro}), conventional single trial
fMRI analysis relies heavily on preexisting knowledge of event timing
\citep{Buckner1996Detectioncorticalactivation,Menon1998Mentalchronometryusing,Richter1997TimeresolvedfMRI}.
However, in certain situations such as clinical settings, resting-state, or
naturalistic paradigms, it can be challenging or even impossible to formulate a
temporal model of activations. This thesis has introduced new fMRI analysis
techniques that allow for the study of true single trial BOLD responses without
any prior information about the event timings. These methods greatly improve
upon the \acrlong*{pfm} approach \citep{Gaudes2013Paradigmfreemapping}, and
extend its capabilities to include multivariate and multi-subject settings.

In \cref{cha:synthesis_analysis}, the underlying motivations and principles of
\acrfull*{pfm} and hemodynamic deconvolution are discussed. Additionally, the
chapter also delved into existing fMRI analysis methods that aim to conduct
single trial experiments without the need for precise or null specifications of
the experimental paradigm. Notably, \acrfull*{ta} was highlighted as a leading
deconvolution technique that follows an analysis formulation, in contrast to the
synthesis formulation of PFM. The chapter also introduced the block model for
PFM, which allows for the estimation of the innovation signal --the derivative
of the activity-inducing signal. This approach is especially useful for
block-design experiments and utilizes the sparsity constraint of the LASSO more
effectively. After thorough comparison, it was found that the two methods are
essentially equivalent, with PFM emerging as the preferred option due to its
simplistic and adaptable formulation. This paved the way for the development of
various PFM techniques in this thesis, including stability-based PFM
(\cref{cha:stability}), multivariate PFM (\cref{cha:multivariate}),
multi-subject PFM (\cref{cha:multi-subject}), and sparse and low-rank PFM
(\cref{cha:low-rank}).

A crucial aspect for achieving satisfactory operation of the PFM techniques is
the accurate selection of the regularization parameters, as the deconvolution
relies on regularized estimators. In the original PFM techniques
\citep{Gaudes2013Paradigmfreemapping}, the choice of the regularization
parameter $\lambda$ was determined using Akaike and Bayesian information
criteria independently for each voxel. \cref{cha:stability} introduced an
alternative approach based on stability selection, which eliminates the need for
selecting this parameter altogether. Moreover, this new procedure offers an
additional metric, defined as the area under the curve of the stability path,
that represents the likelihood of the detected events being true at the finest
spatial and temporal scales.

\cref{cha:multivariate} presented a novel approach to the PFM methodology,
introducing a multivariate formulation that organizes voxel or \acrshort*{roi}
time series into a matrix and incorporates the stability selection procedure. By
combining these two techniques with the use of the $\ell_1 + \ell_{2,1}$ mixed
norm group sparsity regularization --which incorporates spatial information and
interactions into the formulation-- the chapter showcased the enhanced
performance of the multivariate PFM with stability selection in comparison to
the original univariate formulation of PFM. Notably, this method yielded results
that were more concordant with single-trial GLM findings. Additionally, the
chapter demonstrated the robustness of the approach across both single-echo and
multi-echo fMRI data. This was achieved through the utilization of the area
under the curve (AUC) measure, which facilitated the application of various
thresholding techniques adaptable to the noise level inherent in the data.

\cref{cha:multi-subject}
introduced a new application of the multivariate PFM formulation: simultaneous
deconvolution of multiple subjects performing a naturalistic paradigm. In this
case, instead of concatenating voxels or ROIs to form a time by space matrix,
the same voxel or ROI from different subjects was used to create a time by
subject matrix. The underlying assumption is that the estimation of
neuronal-related activity should not be significantly affected by anatomical and
functional differences between subjects once their data has been aligned to a
shared space or template. This assumption holds true when analyzing data at the
ROI level, where differences are smoothed out during averaging and spatial
resolution is reduced. However, when working with voxels, this assumption no
longer holds, and differences across subjects are expected. Due to the immense
computational cost and memory requirements of simultaneously deconvolving 43
subjects at the voxel level, the analysis in this chapter was performed at the
ROI level, where the assumption remained valid. The results demonstrated the
capability of the multi-subject PFM (msPFM) algorithm to detect shared and
individualized neuronal-related activity across subjects without prior
knowledge. Furthermore, the technique successfully linked moment-to-moment brain
activity to its underlying causes. Encouragingly, the group synchrony metric
showed significant correlations with changes in luminance, audio, speech, the
presence of hands, and the presence of faces. Importantly, \acrshort*{mspfm} was
able to adapt to different movies. For example, in the movie "Iteration," where
a single character's face was frequently shown, the group synchrony metric
correlated with the presence of faces in left middle temporal gyrus, which is
associated with face familiarity and gaze. In contrast, in the movie "Sherlock,"
where multiple characters appeared and facial identification was more
challenging, the group synchrony metric correlated with changes in the presence
of faces in the fusiform face areas. These findings indicate that participants
may have adapted to certain characteristics of the stimuli, such as luminance
and the presence of faces, and that the msPFM estimates were able to capture
this adaptation. Although these results are promising, many questions still
remain unanswered. For instance, utilizing msPFM could help us understand how
the human brain comprehends complex multiplexed signals and identify the
specific stimulus features that elicit responses. Furthermore, msPFM has the
potential to elucidate the connection between individual differences in these
responses and subsequent memory formation or appraisal of the stimulus.

In \cref{cha:low-rank}, the nuclear norm was employed as an additional
regularization term in the multivariate PFM formulation. This term effectively
tackles the issue of global fluctuations in the BOLD signal, including
motion-related signals and physiological artifacts, which can distort the
deconvolution of neuronal-related activity. By employing the sparse and low-rank
PFM algorithm, the method successfully mitigated this bias, resulting in an
accurate estimation of the activity-inducing signal. Notably, the results
exhibited a remarkable similarity between the sparse and low-rank PFM algorithm
and the single-trial GLM results in detecting neuronal-related activity in a
complex dataset characterized by numerous conditions in the experimental task.
When the regularization parameter $\lambda$ was manually selected, the algorithm
performed comparably to the multivariate PFM. However, the selection of the
regularization parameter for the nuclear norm penalty in the sparse and low-rank
PFM posed a significant challenge, which will be addressed in the subsequent
section. Given the success of the stability selection procedure in
\cref{cha:stability} and \cref{cha:multivariate}, it is worth exploring its
adaptation for this algorithm to avoid selecting all three regularization
parameters.

Overall, these series of studies demonstrate that PFM techniques can be used to
reliably retrieve the neuronal-related activity from fMRI data without any prior
information about the experimental paradigm, and that there now exists a
formulation of PFM that is suitable for potentially any experimental setting and
research question.

Finally, the code used for all the research and algorithms presented in this
thesis was written in Python. To promote the use of the PFM techniques developed
within this thesis, three separate Python packages have been created and made
available as open source software: \texttt{pySPFM} for the univariate analysis
and as the core library for the other two, \texttt{splora} for the multivariate
and sparse and low-rank PFM, and \texttt{msPFM} for the multi-subject version of
PFM. The packages are available on GitHub at
\url{https://github.com/paradigm-free-mapping} and can be installed using the
Python package manager \texttt{pip}.

\section{Future Developments}

This thesis showcases research findings that emphasize certain aspects deserving
further developments or refinement. First of all, this thesis assumed an
identical hemodynamic response model for the entire brain. However, the waveform
of the hemodynamic response function (HRF) is known to vary across voxels within
cortical regions, across cortical regions, and across subjects
\citep{Aguirre1998VariabilityHumanBOLD,Handwerker2004VariationBOLDhemodynamic,Miezin2000CharacterizingHemodynamicResponse,Zwart2005TemporaldynamicsBOLD,Saad2001AnalysisuseFMRI},
potentially reflecting different local distributions of vascular anatomy and
neurovascular coupling. Therefore, employing prior information about the HRF
would allow a more precise estimation of neuronal-related events. Its
implementation is straightforward in the case of the univariate formulation of
PFM. However, the use of voxel- or region-specific HRFs is not as
straightforward in a multivariate scenario. An alternative solution would be to
apply the multivariate PFM technique within ROIs using an ROI-specific HRF. An
alternative strategy would be to adaptively model the HRF by including the
temporal and dispersion derivatives of the assumed canonical HRF in the Toeplitz
matrix of the deconvolution model
\citep{Gaudes2012Structuredsparsedeconvolution}.

Another area of focus to improve PFM is the integration of new regularization
terms to mitigate the bias associated with the $\ell_1$-norm. In cases where a
grouping sparsity constraint is not applied or when dealing with the univariate
PFM, an effective method would be to implement regularization with the
$\ell_0$-norm. However, this presents a challenge as the optimization problem
associated with the $\ell_0$-norm is known to be a non-convex, NP hard problem.
In such cases, an alternative solution could be to utilize $\ell_{0.5}$-norm
regularization instead. For the multivariate formulations, a viable option would
be to adopt the OSCAR (octagonal selection and clustering algorithm for
regression) regularizer. This method involves the use of a combination of
$\ell_1$ and pair-wise $\ell_{\infty}$-norms, which is responsible for its
grouping behavior. This approach was proposed to promote group sparsity in
situations where the groups are not known beforehand
\citep{Bondell2008SimultaneousRegressionShrinkage,Gueddari2021CalibrationLessMulti}.

The multivariate formulation of PFM has introduced a significant advancement by
enabling the incorporation of spatial information and interactions into the
estimation of neuronal-related activity. This opens up possibilities for
enhancing the estimation process. For example, the matrix representing the
estimated activity-inducing signal $\mathbf{S}$ could be multiplied by a
connectivity matrix that represents the interactions between voxels or ROIs in
the penalty term. This connectivity matrix could be obtained from other imaging
modalities, either structural connectivity from diffusion-weighted MRI or
functional connectivity from complementary electrophysiological recordings
(e.g., EEG or MEG). 

The main limitation of the current multi-subject PFM approach is the assumption
that the anatomical and hemodynamic differences between subjects do not
significantly affect the estimation of neuronal-related activity when all the
data is moved into a shared space. This assumption holds to some extent when
working with ROIs, where spatial resolution is reduced and finer anatomical
differences across subjects are smoothed out during averaging. However, this
assumption no longer holds striclty in the case of operating at the voxel level.
In other words, a perfect anatomical alignment and voxel-to-voxel correspondence
in the location of activations cannot be assumed across different subjects. To
address the issue of inter-subject spatial variability of functional
activations, optimal transport theory could be adopted. This approach, as
demonstrated in
\citep{Gramfort2015FastOptimalTransport,Janati2019Wassersteinregularizationsparse,Janati2020MultisubjectMEG/EEG},
does not require exact spatial correspondence between neuronal-related events in
the group of subjects. Instead, it compares the estimates by considering the
geodesic distances between their locations. Furthermore, to tackle the sources
of invariance arising from shifts in time, space, and total population size, a
more sophisticated formulation that integrates dynamic time warping and
unbalanced optimal transport could be considered
\citep{Janati2022AveragingSpatiotemporal}.

Furthermore, the selection of the regularization parameter for the nuclear norm
in the sparse and low-rank PFM poses a significant challenge. In this thesis, a
fixed number of low rank components was chosen as the criterion for selection.
However, if functional activations are widespread, this approach may mistakenly
classify components resembling BOLD responses as a low-rank, global components.
Alternatively, one could consider using a more liberal value for the
regularization parameter and then employ a decision tree to distinguish between
global and neuronal-related components. For instance, such a decision tree could
be developed by drawing inspiration from the ICA AROMA
\citep{Pruim2015ICAAROMArobust} or ICA FIX
\citep{SalimiKhorshidi2014Automaticdenoisingfunctional} methods. Moreover, the
application of the low-rank and sparse PFM to resting-state fMRI data remains to
be studied.

Future research should also consider the development of deep learning methods
for fMRI deconvolution based on physical models of the BOLD signal.
Convolutional neural networks (CNNs) have gained significant attention due to
their exceptional performance in object classification and segmentation tasks,
achieved through training on large image databases. Inspired by these
achievements, \acrshort*{cnn}s have been applied to various inverse problems in
imaging, such as denoising, deconvolution, superresolution, and medical image
reconstruction
\citep{McCann2017ConvolutionalNeuralNetworks,Wang2020Multiresolutionconvolutional}.
These applications have demonstrated promising results, surpassing
state-of-the-art techniques, including compressed sensing. Consequently, the
implementation of CNNs in PFM techniques has the potential to enhance the
accuracy of neuronal-related activity estimates. However, these models are very
expensive to train as they require large amounts of data and computational
power. Hence, physics-based and self-supervising models could be explored for
the development of PFM's deep learning counterparts
\citep{Lucas2018UsingDeepNeural,LopezTapia2021Deeplearningapproaches,Hammernik2023PhysicsDrivenDeep,Aggarwal2019MoDLModelBased}.

Finally, the availability of the Python libraries for researchers is crucial.
However, in order to promote wider adoption of the PFM techniques, it is
essential that these libraries are well documented and include numerous usage
examples. Therefore, the development of a comprehensive documentation and
tutorial is a priority for the near future.
