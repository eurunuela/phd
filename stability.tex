% !TEX root = ../main.tex
\title{Stability-Based Sparse Paradigm Free Mapping}
\label{cha:stability}

\begin{framed}\noindent This chapter was published as
    \fullcite{Urunuela2020StabilityBasedSparse}. DOI:
    \url{10.1109/EMBC44109.2020.9176137}.
\end{framed}

This chapter proposes the implementation of a subsampling approach based on
stability selection that avoids the choice of any regularization parameter for
hemodynamic deconvolution with sparity-promoting regularized least squares
estimators. The proposed method is implemented to operate with the Sparse
\acrlong*{pfm} (SPFM) algorithm and its performance is evaluated on real fMRI
data and compared with both the original SPFM algorithm, which used model
selection criteria to select the parameters, and a conventional analysis with a
general linear model (GLM) that is aware of the temporal model of the
neuronal-related activity. The results demonstrate that SPFM with stability
selection yields activation maps with higher resemblance to the maps obtained
with GLM analyses and offers improved detection of neuronal-related events over
SPFM, particularly in scenarios with low contrast-to-noise ratio.

\section{Introduction}
\label{sec:stability_introduction}

In the preceding chapter, deconvolution approaches were discussed in the context
of functional magnetic resonance imaging (fMRI) data analysis as they offer the
remarkable ability to estimate neuronal-related activity without the need for
prior information on the timings of the blood oxygenation level-dependent (BOLD)
events via a linear time-invariant model (i.e., a forward model of the BOLD
response) that is then inverted by means of regularized least-squares estimators
to deconvolve the neuronal-related activity at each voxel
\citep{Gitelman2003Modelingregionalpsychophysiologic,Khalidov2011ActiveletsWaveletssparse,Karahanoglu2013TotalactivationfMRI,HernandezGarcia2011Neuronaleventdetection,Gaudes2010Detectioncharacterizationsingle,Gaudes2013Paradigmfreemapping}.
In particular, the sparse \acrlong*{pfm} (SPFM) method
\citep{Gaudes2013Paradigmfreemapping}, which is the basis of this work, employs
sparsity-promoting regularization terms based on the L1-norm of the estimates
(e.g., using the LASSO or the Dantzig Selector). Importantly, inverse problem
solving is linked to a dilemma that has yet to be solved: the selection of the
regularization parameters that yield accurate estimates. As described in the
previous chapter,  methods based on model selection criteria after the
computation of the entire regularization path
\citep{Gaudes2013Paradigmfreemapping} or iterative procedures so that the
variance of the residuals after deconvolution is equal to a prior estimate of
the noise variance \citep{Karahanoglu2013TotalactivationfMRI} have been
previously used in the literature for parameter tuning due to their reduced
computational cost. Yet, these methods offer no information about the
appropriateness of the selected parameters.

This chapter proposes the use of the subsampling approach of stability selection
\citep{Meinshausen2010Stabilityselection} with the SPFM algorithm
\citep{Gaudes2013Paradigmfreemapping} to avoid the choice of any regularization
parameter and account for the likelihood of the different possible estimates in
the regularization path. Although stability selection has been previously
proposed in fMRI data analysis, for example in the estimation of functional
connectivity matrices from partial correlations with sparse estimators
\citep{Ryali2012Estimationfunctionalconnectivity} and to detect change points in
time-varying functional connectivity with the graphical lasso
\citep{Cribben2013Detectingfunctionalconnectivity}, its application for the
deconvolution of the fMRI signal is innovative. Further, this chapter implements
a novel procedure that enables to benefit from the computational speed of the
least angle regression algorithm \citep{Efron2004Leastangleregression} in
combination with the robustness of stability selection. 

This chapter uses a modification of the original SPFM formulation called
\textit{block model} --and introduced in \cref{cha:synthesis_analysis}-- that
computes estimates of the innovation signal of the neuronal-related signal
(i.e., defining its changes)
\citep{Karahanoglu2013TotalactivationfMRI,Cherkaoui2019Sparsitybasedblind,Urunuela2023HemodynamicDeconvolutionDemystified},
rather than the signal itself. The block model formulation fits the data used in
this study better as it improves the estimation of neuronal-related events with
long, sustained activity
\citep{Karahanoglu2013TotalactivationfMRI,Cherkaoui2019Sparsitybasedblind,Urunuela2023HemodynamicDeconvolutionDemystified}
that cannot be adequately described by conventional spike-like models
\citep{Khalidov2011ActiveletsWaveletssparse,Gaudes2010Detectioncharacterizationsingle,Gaudes2013Paradigmfreemapping}.
Nevertheless, the proposed stability selection procedure can be readily
implemented for the spike model. The chapter is organized as follows: in
\cref{sec:stability_model} the signal model and the stability-based SFPM
algorithm are introduced; in \cref{sec:stability_methods}, the results of
applying this new algorithm on experimental fMRI data are presented and compare
them to the previous SPFM algorithm.

\section{Signal Model and Deconvolution with Stability-Based \acrlong*{pfm}}
\label{sec:stability_model}
For the sake of completeness, this section revises the signal model for hemodynamic deconvolution, which was already described in \cref{cha:synthesis_analysis}. 
In fMRI data analysis, the signal of a voxel $y(t)$ is commonly modelled as the
convolution of an underlying neuronal-related signal $s(t)$ with the hemodynamic
response function (HRF) $h(t)$, plus a white noise component: $y(t)=
h(t)*s(t)+n(t)$, or $\mathbf{y}=\mathbf{Hs}+\mathbf{n}$ in a discrete-time matrix
notation. Typically, the neuronal-related signal $s(t)$ is represented as a
train of Dirac impulses at the fMRI timescale associated with the experimental
design. This model of the neuronal-related signal has been adopted by previous
deconvolution algorithms
\citep{HernandezGarcia2011Neuronaleventdetection,Gaudes2010Detectioncharacterizationsingle,Gaudes2013Paradigmfreemapping}
relying on regularized least-squares estimators as follows:

\begin{equation}
    \mathbf{\hat{s}} = \arg \min_{\mathbf{s}} \frac{1}{2} \|\mathbf{y} - \mathbf{Hs}\|_2^2 + \lambda \|\mathbf{s}\|_p,
    \label{eq:stability_inverse_problem_spike}
\end{equation}
where the $L_p$-norm $\| \mathbf{s} \|_p$ penalizes the amplitude of the
coefficients of the neuronal-related signal, e.g., $p = 2$ (i.e., ridge
regression) and $p = 1$ (i.e., LASSO) were employed in
\citep{Gaudes2010Detectioncharacterizationsingle} and
\citep{Gaudes2013Paradigmfreemapping}, respectively. Instead of the on/off
pattern described by Dirac impulses, the neuronal-related signal $\mathbf{s}$
can also be represented as a piecewise constant signal in terms of its
innovation signal $\mathbf{u}$ (i.e., its first derivative in time). Defining
$\mathbf{s}=\mathbf{Lu}$ where $\mathbf{L}$ corresponds to the discrete
integration operator \citep{Cherkaoui2019Sparsitybasedblind}, the signal model
can be written as:

\begin{equation}
    \mathbf{y} = \mathbf{HLu} + \mathbf{n},
    \label{eq:stability_signal_model}
\end{equation}
where $\mathbf{y}, \mathbf{u}, \mathbf{s}, \mathbf{n} \in \mathbb{R}^N$, $L \in
\mathbb{R}^{NxN}$, and $\mathbf{H} \in \mathbb{R}^{NxN}$ is the Toeplitz convolution
matrix with shifted HRFs, where $N$ is the number of observations of the fMRI
signal. The signal $\mathbf{u}$ will represent those instances when significant
changes in the neuronal-related activity occur. Since the innovation signal
$\mathbf{u}$ is sparser than the neuronal-related signal $\mathbf{s}$, it is
also a more adequate representation if the temporal deconvolution of the fMRI
time series of each voxel is performed with L1-norm regularized estimators as
follows:

\begin{equation}
    \mathbf{\hat{u}} = \arg \min_{\mathbf{u}} \frac{1}{2} \|\mathbf{y} - \mathbf{HLu}\|_2^2 + \lambda \|\mathbf{u}\|_1.
    \label{eq:stability_inverse_problem_innovation}
\end{equation}


\subsection{Combining Stability Selection with Least Angle Regression}

An appropriate choice of the regularization parameter $\lambda$ in
\cref{eq:stability_inverse_problem_spike} or
\cref{eq:stability_inverse_problem_innovation} is crucial for appropriate hemodynamic deconvolution. Several techniques to select it have already been proposed, such as based on the
Bayesian Information Criterion \citep{Gaudes2013Paradigmfreemapping}. However,
these techniques do not provide a solution that is robust regardless
of the different characteristics the data may show (e.g., signal-to-noise ratio,
occurrence and duration of neuronal events). 

This problem can be overcome by implementing a novel procedure based on the stability selection approach \citep{Meinshausen2010Stabilityselection}. This procedure generates $T = 100$ surrogate datasets $y_i  (i=1,\dots,T)$ where the original voxel time series is randomly subsampled to retain 60\% of its time points. Then, the optimization problem in \cref{eq:stability_inverse_problem_innovation} is solved for each surrogate dataset, where the model matrix $\mathbf{H}$ is
subsampled accordingly. Then, the stability paths of the signal $\mathbf{u}$ for
each surrogate $i$ and each time point $t$ (i.e., $u_t^i$) are computed, which
represent the probability of the coefficient being non-zero for a given
$\lambda$. Originally, the stability selection approach operates by solving
\cref{eq:stability_inverse_problem_innovation} for a predefined set of $\lambda$
values, for example by means of the fast iterative shrinkage thresholding
algorithm (FISTA) \citep{Beck2009FastIterativeShrinkage}. Instead, this
chapter proposes to use the least angle regression (LARS) algorithm
\citep{Efron2004Leastangleregression}, which computes the entire regularization
path for an optimal decreasing set of $\lambda$ values and is faster than
\acrshort*{fista} \citep{Beck2009FastIterativeShrinkage} for our purposes. Then,
for each surrogate, the estimate $u_(\lambda_l,t)^i$ at the regularization
parameter $\lambda_l$ and time point $t$ is binarized as $c_(\lambda_l,t)^i=0$
if $u_(\lambda_l,t)^i=0$ or $c_(\lambda_l,t)^i=1$ otherwise. To overcome the
fact that solving \cref{eq:stability_inverse_problem_innovation} with the LARS
algorithm will generate a different set of $\lambda$ values in each subsampled
surrogate, a new set of $\lambda$ values is created. This new set contains all
of the regularization parameters from all of the surrogate-specific
regularization paths in decreasing order. The
coefficients $c_(\lambda_l,t)^i$ remain 0 or 1 according to the preceding value
of $\lambda_l^i$ corresponding to the surrogate-specific regularization path
computed by \acrshort*{lars}. This step allows us to calculate the probabilities
that construct the stability paths as the ratio of surrogates where each
coefficient $u_(\lambda,t)$ is different from 0 at each $\lambda$. 

Furthermore, unlike in the original stability selection procedure that sets a given
probability threshold to select the final set of non-zero coefficients, this implementation
calculates the area under the curve (AUC) of the stability paths of each coefficient $u_t$ as follows:

\begin{equation}
    \text{AUC}_t = \frac{\sum_{l=1}^L \lambda_l P\left( u_{\lambda_l,t} \neq 0 \right)}{\sum_{l=1}^L \lambda_l},
    \label{eq:stability_auc}
\end{equation}
where 
\begin{equation}
    P\left( u_{\lambda_l,t} \neq 0 \right) = T^{-1} \sum_{i=1}^T c_{\lambda_l,t}^i,
    \label{eq:stability_auc_2}
\end{equation} 
represents the selection probability of coefficient
$u_(\lambda,t)$ for a particular choice of the regularization parameter
$\lambda_l$, and $L$ is the total number of regularization parameters from all
of the LARS regularization paths. Hence, the voxelwise time series
$\text{AUC}(t)$ reveals the most prominent coefficients indicating the
probability of activation at each time point.

\subsection{Thresholding and Debiasing}

Afterwards, the AUC time series for each voxel are thresholded to identify those
instances with high probability of a significant change in neuronal-related
activity occurring. This threshold is based on a given percentile (or maximum)
of the AUC values in a region of interest where no BOLD signal changes related
to neuronal activity are assumed to occur (or can be detected). For example,
the threshold can be set to the 99th percentile of the AUC values of deep white
matter (DWM) voxels (see \cref{sec:stability_results}). 

Finally, it is recommended to remove the bias in the estimates of the neuronal-related signal
owing to the L1-norm regularization term. For the signal model in
\cref{eq:stability_inverse_problem_spike} used in the original SPFM approach
\citep{Gaudes2013Paradigmfreemapping}, a debiased estimate of $\mathbf{s}$ can
be obtained by solving a least squares problem with a selection of
non-zero AUC coefficients.

Rather, in the signal model with the innovation signal, the selected non-zero
coefficients of $\mathbf{u}$ are used to define a matrix $\mathbf{A}$ whose
columns are activation segments with piecewise constant unit between two
non-zero coefficients of $\mathbf{u}$ ~\citep{Zoller2019RobustRecoveryTemporal}.
A final debiased estimate of $\mathbf{s}$ is obtained by solving the following
least squares problem:

\begin{equation}
    \mathbf{\hat{u}} = \arg \min_{\mathbf{u}} \|\mathbf{y} - \mathbf{HAu}\|_2^2
    \label{eq:stability_debiasing_innovation}
\end{equation}

\begin{figure}[!ht]
    \centering
    \includegraphics[width=0.8\textwidth]{figures/stability/pipeline.jpg}
    \caption{Flowchart of the stability-based SPFM algorithm.}
    \label{fig:stability_flowchart}
\end{figure}

\cref{fig:stability_flowchart} illustrates the flowchart of the proposed stability-based SPFM algorithm.

\section{Methods}
\label{sec:stability_methods}

The operation of the proposed stability-selection SPFM algorithm is illustrated
in a dataset collected on a healthy subject in a 7T MR scanner (Siemens) using a
32-channel receive transmit coil. The subject was scanned under a Cleveland
Clinic Institutional Review Board approved protocol (QED, Cleveland, OH). A
volumetric MP2RAGE image was acquired for anatomical visualization. Two fMRI
datasets were acquired with a simultaneous multislice EPI sequence (MB factor =
3, TE = 21 ms, field of view = $192 \times 192 \; \text{mm}^2$) at TR = 2800 ms
($1.2 \times 1.2 \times 1.5 \; \text{mm}^3$, flip angle = 55º) and 500 ms ($3
\times 3 \times 3 \; \text{mm}^3$, flip angle = 70º). For both TRs, the subject
performed finger tapping events with the right index and thumb fingers every 45
s, where a single tap was performed in the first 6 minutes, or 10 taps quickly
for the remaining 4 minutes. The onsets and durations of the paradigm are shown
as gray vertical lines in \cref{fig:stability_results_a} (a) and
\cref{fig:stability_results_b} (a).

Data preprocessing comprised an initial correction for motion using SLOMOCO2
\citep{Beall2014SimPACEGeneratingsimulated}, detrending of 6th order Legendre
polynomials and normalization to signal percentage change (SPC) with AFNI.
Furthermore, a mask of white matter voxels was computed from the anatomical
image with 3dSeg, which was then eroded 2 voxels to delimit voxels in deep white
matter in the functional space. The preprocessed fMRI data were analyzed with
three different methods: 1) a traditional general linear model (GLM) analysis
using the onsets and durations of the tapping events; 2) the original SPFM
approach (3dPFM) using the LASSO for deconvolution and selection of the
regularization parameter based on the Bayesian Information Criterion (BIC)
\citep{Gaudes2013Paradigmfreemapping}; and 3) the novel stability-based SPFM
with and without the integration operator in its formulation. Both SPFM
approaches used the double-gamma canonical HRF as a model for deconvolution
(SPMG1 shape in 3dDeconvolve in AFNI). Previous to the final debiasing step,
spatio-temporal clustering of a minimum of 5 contiguous voxels with activation
(i.e., non-zero coefficient after thresholding) in a temporal window of $\pm 1$
TR was also performed to remove spurious, scattered activations.

\section{Results and Discussion}
\label{sec:stability_results}

\begin{figure}[!ht]
    \centering
    \includegraphics[width=0.95\textwidth]{figures/stability/stability_results_a.jpg}
    \caption{Comparison of the novel stability-based SPFM approach with the SPFM
    and the GLM methods for data with TR = 0.5 s and a voxel size of $3 {mm}^3$
    iso. (a) plots the time series of the voxels marked with a cross in (c)
    containing the raw data and the estimates of the different methods as shown
    in the legend. Onsets and duration of the finger-tapping are depicted as
    gray vertical lines. (b) shows the estimates of the different methods fitted
    with the canonical HRF. (c) shows the estimated maps of each of the methods
    for different finger-tapping events.}
    \label{fig:stability_results_a}
\end{figure}

\begin{figure}[!ht]
    \centering
    \includegraphics[width=0.95\textwidth]{figures/stability/stability_results_b.jpg}
    \caption{Comparison of the novel stability-based SPFM approach with the SPFM
    and the GLM methods for data with TR = 2.8 s and a voxel size of
    $1.2\times1.2\times1.5 {mm}^3$. (a) plots the time series of the voxels
    marked with a cross in (c) containing the raw data and the estimates of the
    different methods as shown in the legend. Onsets and duration of the
    finger-tapping are depicted as gray vertical lines. (b) shows the estimates
    of the different methods fitted with the canonical HRF. (c) shows the
    estimated maps of each of the methods for different finger-tapping events.}
    \label{fig:stability_results_b}
\end{figure}

\cref{fig:stability_results_a} and \cref{fig:stability_results_b} depict the
activity maps estimated with all of the methods for different representative
finger-tapping instants and the time courses of a voxel in the left primary
motor cortex (marked with a white cross in the maps) for the high temporal and
low spatial resolution dataset \cref{fig:stability_results_a} (a, b and c) and
the low temporal and high spatial resolution dataset
\cref{fig:stability_results_b} (a, b and c).

In the high temporal and low spatial resolution scenario (i.e., a high
contrast-to-noise ratio regime), the activity maps in
\cref{fig:stability_results_a} (c) illustrate that the original SPFM is able to
detect finger tapping events with a high specificity. Implementing stability
selection on the original SPFM algorithm increases the sensitivity while
maintaining the specificity. However, as it can be seen in
\cref{fig:stability_results_a} (a), the lack of an integration operator yields
very variable estimates of the neuronal-related signal after debiasing with
least squares (here scaled by 0.05 for visualization purposes) due to the large
correlation of the debiasing model with contiguous non-zero coefficients at this
fast temporal resolution. Conversely, the novel stability-based SPFM with the
integration operator shows activity maps that are comparable to the ground truth
despite the lower amplitude of the estimates. Yet, it can be observed that the
signal model with the integrator overestimates the duration of the piecewise
constant estimates for the short finger tapping events. Thus, in this scenario,
the use of the stability selection and the innovation signal exhibits a similar
performance to the original SPFM algorithm using LASSO and BIC since the high
SNR and high temporal resolution (TR = 0.5 s) enables a precise and clear
characterization of the dynamics of the BOLD signal, which facilitates the
differentiation between noise and neuronal-related signal.

In an acquisition with a high spatial resolution and a low temporal resolution
(i.e., a low contrast-to-noise ratio regime), \cref{fig:stability_results_b}
(a), (b) and (c) demonstrate that the novel stability-based SPFM approach is
able to detect more finger-tapping events and their associated brain activity
than the original SPFM method. This advantage is clearly seen in the case of the
single-tapping events, which exhibit a lower amplitude in the response than the
long events with ten consecutive finger taps. The stability selection proves to
be essential in correctly estimating finger tapping events, regardless of the
use of the integration operator. The addition of the integration operator to the
SPFM model produces activity maps that are closer to the ground truth of the GLM
analysis (see \cref{fig:stability_results_a} c), even though the duration of
the piecewise constant estimates is overestimated (see
\cref{fig:stability_results_b} a). In this regime, the BIC criterion in the
original SPFM is not able to discern between neuronal-related events and noise,
failing to detect the finger tapping events, probably as the shape of the BOLD
response, which takes 4-6 s to reach its maximum amplitude, cannot be properly
characterized by the model owing to the low temporal resolution (TR = 2.8 s).
Hence, the stability selection procedure exhibits a robust performance at
correctly estimating the neuronal-related events resulting from the finger
tapping tasks, which showcases that the additions to the SPFM technique are
promising, especially in low temporal resolution settings.